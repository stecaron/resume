
\cvsection{Éducation}
\cvevent{Maîtrise en statistique avec mémoire}{Université Laval}{Sep 2017 -- Déc 2020}{}
\cvevent{Baccalauréat en actuariat}{Université Laval}{Sep 2012 -- Mai 2015}{}
% \cvachievement{\faTrophy}{}{Received accolades at Atos for Best Performance in team.}
% \cvachievement{\faTrophy}{}{Received Best Debut Award at Atos. }
% %\divider
% \cvachievement{\faInstitution}{}{Won 2nd Consolation Prize for paper presented on Cognitive Radio Networks.}
% %\divider
% \cvachievement{\faGraduationCap}{}{Got Selected in "Exclusive Scholar Program" during undergrad.}
% %\divider
% \cvachievement{\faDollar}{}{Awarded with Narotam Sekhsaria Foundation Scholarship}
%\cvsection{Strengths}

%\cvtag{Hard-working (18/24)} 
%\cvtag{Persuasive}
%\cvtag{Motivator \& Leader}

%\divider\smallskip

%\cvtag{UX}
%\cvtag{Mobile Devices \& Applications}
%\cvtag{Product Management \& Marketing}


%\divider

%\cvevent{B.S.\ in Symbolic Systems}{Stanford University}{Sept 1993 -- June 1997}{}

\cvsection{Projets}
\cvproject{Détection d'anomalies basée sur les représentations latentes d'un autoencodeur variationnel}
\begin{itemize}
\item Dans mon mémoire de maîtrise, j'ai utilisé la représentation latente d'un VAE pour identifier les images anormales d'un jeu de données. J'ai également comparé les résultats avec d'autres approches traditionnelles de détection d'anomalies sur des images. \\ \href{https://github.com/stecaron/deep-stat-thesis/blob/master/rapports/main-memoire.pdf}{En savoir plus}.
\end{itemize}
\smallskip
\cvproject{Segmentation sémantique sur des images de matchs de hockey}
\begin{itemize}
\item Dans ce projet d'équipe, nous avons conçu et entraîné un réseau de neurones à convolution (CNN) capable d'apprendre la sémantique d'une image de match de hockey. Nous avons annoté nous-mêmes les images et avons entraîné un modèle utilisant l'architecture U-Net afin de prédire une classe pour chacun des pixels de l'image. Nous avons également présenté ce projet durant la session de \textit{posters} à l'événement 2019 Ottawa Hockey Analytics Conference. \\ \href{https://github.com/stecaron/glo-7030-projet/blob/master/pancarte/main-pancarte.pdf}{En savoir plus}.
\end{itemize}
\smallskip
\cvproject{Meetup ML en assurance}
\begin{itemize}
\item J'ai contribué à l'organisation d'un meetup en apprentissage automatique et j'étais aussi responsable de définir la problématique proposée durant la section hackathon de l'événement. Le problème était d'identifier des toits verts à partir d'images satellites de toits d'immeubles.
\end{itemize}
\smallskip
\cvproject{Big Data Cup 2021: Ottawa Hockey Analytics Conference}
\begin{itemize}
\item Dans cette compétition, nous avons proposé une analyse sur l'impact de remporter une mise au jeu en zone offensive durant un match de hockey. Nous avons participé dans la catégorie "Open".  \\ \href{https://github.com/dot-layer/bigdatacup-2021/blob/main/report/CaronLeCavalierPerreault-BigDataCup2021.pdf}{En savoir plus}.
\end{itemize}
\smallskip
\cvproject{\textit{.Layer} blog website}
\begin{itemize}
\item J'ai contribué à la création du site web de l'OSBL \textit{.Layer}. J'ai également contribué en rédigeant plusieurs articles de blog et en révisant d'autres articles. \\ \href{https://www.dotlayer.org/}{En savoir plus}.
\end{itemize}
\smallskip
