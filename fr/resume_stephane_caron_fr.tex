\documentclass[10pt,a4paper,ragged2e]{altacv}
\geometry{left=2cm,right=10cm,marginparwidth=6.8cm,marginparsep=1.2cm,top=1.25cm,bottom=1.25cm}
\ifxetexorluatex
  \setmainfont{Carlito}
\else
  \usepackage[utf8]{inputenc}
  \usepackage[T1]{fontenc}
  \usepackage[default]{lato}
\fi
\definecolor{VividPurple}{HTML}{000000}
\definecolor{SlateGrey}{HTML}{2E2E2E}
\definecolor{LightGrey}{HTML}{2E2E2E}
\colorlet{heading}{VividPurple}
\colorlet{accent}{VividPurple}
\colorlet{emphasis}{SlateGrey}
\colorlet{body}{LightGrey}
\renewcommand{\itemmarker}{{\small\textbullet}}
\renewcommand{\ratingmarker}{\faCircle}
\addbibresource{sample.bib}

\usepackage{hyperref}
\usepackage{xcolor}
\hypersetup{
	colorlinks,
	linkcolor={red!50!black},
	citecolor={blue!50!black},
	urlcolor={blue!80!black}
}
\usepackage{fontawesome}


\begin{document}
\name{STÉPHANE CARON}
\tagline{Scientifique de données}
% Cropped to square from https://en.wikipedia.org/wiki/Marissa_Mayer#/media/File:Marissa_Mayer_May_2014_(cropped).jpg, CC-BY 2.0
%\photo{3.3cm}{profile.jpg}
\personalinfo{%
  % Not all of these are required!
  % You can add your own with \printinfo{symbol}{detail}
  \email{ste.caron@icloud.com}
  \phone{581 997-5006}
%  \mailaddress{Address, Street, 00000 County}
  \location{Montréal, Canada}
%  \homepage{marissamayr.tumblr.com/}
%  \twitter{@marissamayer}
  \linkedin{\href{https://www.linkedin.com/in/ste-caron/}{linkedin.com/in/ste-caron/}}
  \github{\href{https://github.com/stecaron}{github.com/stecaron}}
%   \orcid{orcid.org/0000-0000-0000-0000} % Obviously making this up too. If you want to use this field (and also other academicons symbols), add "academicons" option to \documentclass{altacv}
}

%% Make the header extend all the way to the right, if you want.
\begin{fullwidth}
\makecvheader
\end{fullwidth}

%% Depending on your tastes, you may want to make fonts of itemize environments slightly smaller
\AtBeginEnvironment{itemize}{\small}

%% Provide the file name containing the sidebar contents as an optional parameter to \cvsection.
%% You can always just use \marginpar{...} if you do
%% not need to align the top of the contents to any
%% \cvsection title in the "main" bar.
\cvsection[page1sidebar]{Expérience}

\cvevent{Scientifique de données principal}{Intact Assurance}{Juin 2019 -- Aujourd'hui}{Montréal, Canada}
\begin{itemize}
	\item Travaillé sur différents projets utilisant les photos de véhicules accidentés (estimation de dommages, détermination de perte total, etc).
	\item Utilisé des regréssions linéaires et des séries temporelles pour développer un modèle de réserves permettant de prédire la valeur ultime d'une réclamation.
	\item Impliqué dans le processus d'intrevues des stagiaires en science de données.
\end{itemize}

\smallskip

\cvevent{Analyste en actuariat}{Intact Assurance}{Mar 2018 - Juin 2019}{Québec, Canada}
\begin{itemize}
	\item Développé un modèle de détection de fraudes permettant de prioriser les réclamations à investiguer.
\end{itemize}

\divider

\cvevent{Analyste en actuariat}{Promutuel Assurance}{Mai 2015 - Mar 2018}{Québec, Canada}
\begin{itemize}
	\item Développé plusieurs rapports dynamiques sur les performances d'assurance utilisant des outils comme \texttt{RMarkdown} et \texttt{shiny}.
	\item Créé plusieurs programmes d'extraction de données.
	\item Travaillé sur le développement et l'intégration du premier produit de télématique dans l'entreprise.
\end{itemize}

\cvsection{Réalisations}
\smallskip
\begin{itemize}
	\item Associé de la Casualty Actuarial Society (ACAS).
	\item École d'hiver IVADO/MILA en apprentissage profond (2018).
	\item Président de \href{https://www.dotlayer.org/about/}{.Layer}, une OSBL impliquée dans la communauté d'apprentissage automatique de Québec.
	\item Vice-Président Sports de l'association des étudiants en actuariat de l'Université Laval (2 années).
\end{itemize}

\cvsection{COMPÉTENCES TECHNIQUES}
\smallskip
\begin{itemize}
	\item R, Python, Git, LaTeX, SQL, Excel, VBA, SAS
	\item Linux, AWS Cloud Computing (débutant), Docker (débutant)
\end{itemize}

\cvsection{COMPÉTENCES PERSONNELLES}
\smallskip
\begin{itemize}
	\item Qualités de leadership et bon joueur d'équipe.
	\item Habilité à initier des projets de recherche et développement.
\end{itemize}

\cvsection{PASSE-TEMPS}
\smallskip
\begin{itemize}
\item Pêcher, chasser et faire des randonnées en forêt.
\item Voyager et découvrir de nouveaux endroits.
\item Cuisiner et goûter de nouveaux plats.
\end{itemize}

% \cvevent{Product Engineer}{Google}{23 June 1999 -- 2001}{Palo Alto, CA}

% \begin{itemize}
% \item Joined the company as employe \#20 and female employee \#1
% \item Developed targeted advertisement in order to use user's search queries and show them related ads
% \end{itemize}

%\cvsection{A Day of My Life}

% Adapted from @Jake's answer from http://tex.stackexchange.com/a/82729/226
% \wheelchart{outer radius}{inner radius}{
% comma-separated list of value/text width/color/detail}
% Some ad-hoc tweaking to adjust the labels so that they don't overlap
% \wheelchart{1.5cm}{0.5cm}{%
%   10/10em/accent!30/Sleeping \& dreaming about work,
%   25/9em/accent!60/Public resolving issues with Yahoo!\ investors,
%   5/13em/accent!10/\footnotesize\\[1ex]New York \& San Francisco Ballet Jawbone board member,
%   20/15em/accent!40/Spending time with family,
%   5/8em/accent!20/\footnotesize Business development for Yahoo!\ after the Verizon acquisition,
%   30/9em/accent/Showing Yahoo!\ employees that their work has meaning,
%   5/8em/accent!20/Baking cupcakes
% }

\clearpage

% \cvsection[page2sidebar]{Publications}

\nocite{*}

% \printbibliography[heading=pubtype,title={\printinfo{\faBook}{Books}},type=book]

% \divider

% \printbibliography[heading=pubtype,title={\printinfo{\faFileTextO}{Journal Articles}}, type=article]

% \divider

% \printbibliography[heading=pubtype,title={\printinfo{\faGroup}{Conference Proceedings}},type=inproceedings]

% %% If the NEXT page doesn't start with a \cvsection but you'd
% %% still like to add a sidebar, then use this command on THIS
% %% page to add it. The optional argument lets you pull up the
% %% sidebar a bit so that it looks aligned with the top of the
% %% main column.
% % \addnextpagesidebar[-1ex]{page3sidebar}


\end{document}

